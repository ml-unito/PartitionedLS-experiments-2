\documentclass[dvipsnames]{article}
\usepackage{pgfplots}
\usepackage{subcaption}

\begin{document}

\pgfplotsset{yticklabel style={text width=3em,align=right}}

\begin{figure}
\centering
\begin{subfigure}{0.49\textwidth}
\begin{tikzpicture}
\begin{axis}[
    scale=0.7,
    xmode=log,
    xmin=0, xmax=10,
    grid=both,
    major grid style={black!50}
]
\addplot[orange, mark=*] table [x=TimeCumulative, y=TrainBest, col sep=comma]{experiments/time-vs-obj/Limpet/results-ALT-T20.csv};
\addplot[NavyBlue, mark=*] table [x=TimeCumulative, y=TrainBest, col sep=comma]{experiments/time-vs-obj/Limpet/results-ALT-T100.csv};
\addplot[OrangeRed, mark=*] table [x=TimeCumulative, y=TrainBest, col sep=comma]{experiments/time-vs-obj/Limpet/results-OPT.csv};
\addplot[Green, mark=*] table [x=TimeCumulative, y=TrainBest, col sep=comma]{experiments/time-vs-obj/Limpet/results-BnB.csv};
\addlegendentry{PartLS-Alt T20}
\addlegendentry{PartLS-Alt T100}
\addlegendentry{PartLS-Opt}
\addlegendentry{PartLS-BnB}

\end{axis}
\end{tikzpicture}
\caption{Limpet}
\label{fig:limpet}
\end{subfigure}
\begin{subfigure}{0.49\textwidth}

\begin{tikzpicture}
\begin{axis}[
    scale=0.7,
    xmode=log,
    % xmin=0, xmax=1e1,
    grid=both,
    major grid style={black!50}
]
\addplot[orange, mark=*] table [x=TimeCumulative, y=TrainBest, col sep=comma]{experiments/time-vs-obj/Superconductivty Data/results-ALT-T20.csv};
\addplot[NavyBlue, mark=*] table [x=TimeCumulative, y=TrainBest, col sep=comma]{experiments/time-vs-obj/Superconductivty Data/results-ALT-T100.csv};
\addplot[OrangeRed, mark=*] table [x=TimeCumulative, y=TrainBest, col sep=comma]{experiments/time-vs-obj/Superconductivty Data/results-OPT.csv};
\addplot[Green, mark=*] table [x=TimeCumulative, y=TrainBest, col sep=comma]{experiments/time-vs-obj/Superconductivty Data/results-BnB.csv};

% \addlegendentry{PartLS-Alt T20}
% \addlegendentry{PartLS-Alt T100}
% \addlegendentry{PartLS-Opt}
\end{axis}
\end{tikzpicture}
    
\caption{Superconductivity}
\label{fig:second}
\end{subfigure}
\hfill
\begin{subfigure}{0.49\textwidth}
\begin{tikzpicture}
\begin{axis}[
    scale=0.7,
    xmode=log,
    % xmin=0, xmax=1e1,
    grid=both,
    major grid style={black!50}
]
\addplot[orange, mark=*] table [x=TimeCumulative, y=TrainBest, col sep=comma]{experiments/time-vs-obj/Facebook Comment Volume Dataset/results-ALT-T20.csv};
\addplot[NavyBlue, mark=*] table [x=TimeCumulative, y=TrainBest, col sep=comma]{experiments/time-vs-obj/Facebook Comment Volume Dataset/results-ALT-T100.csv};
\addplot[OrangeRed, mark=*] table [x=TimeCumulative, y=TrainBest, col sep=comma]{experiments/time-vs-obj/Facebook Comment Volume Dataset/results-OPT.csv};
\addplot[Green, mark=*] table [x=TimeCumulative, y=TrainBest, col sep=comma]{experiments/time-vs-obj/Facebook Comment Volume Dataset/results-BnB.csv};

% \addlegendentry{PartLS-Alt T20}
% \addlegendentry{PartLS-Alt T100}
% \addlegendentry{PartLS-Opt}
\end{axis}
\end{tikzpicture}    \caption{Facebook Comment Volume}
    \label{fig:third}
\end{subfigure}
\begin{subfigure}{0.49\textwidth}
\begin{tikzpicture}
\begin{axis}[
    scale=0.7,
    xmode=log,
    % xmin=0, xmax=1e1,
    grid=both,
    major grid style={black!50}
]
\addplot[orange, mark=*] table [x=TimeCumulative, y=TrainBest, col sep=comma]{experiments/time-vs-obj/YearPredictionMSD/results-ALT-T20.csv};
\addplot[NavyBlue, mark=*] table [x=TimeCumulative, y=TrainBest, col sep=comma]{experiments/time-vs-obj/YearPredictionMSD/results-ALT-T100.csv};
\addplot[OrangeRed, mark=*] table [x=TimeCumulative, y=TrainBest, col sep=comma]{experiments/time-vs-obj/YearPredictionMSD/results-OPT.csv};
\addplot[Green, mark=*] table [x=TimeCumulative, y=TrainBest, col sep=comma]{experiments/time-vs-obj/YearPredictionMSD/results-BnB.csv};

% \addlegendentry{PartLS-Alt T20}
% \addlegendentry{PartLS-Alt T100}
% \addlegendentry{PartLS-Opt}
\end{axis}
\end{tikzpicture}
    
    \caption{YearPredictionMSD}
    \label{fig:fourth}
\end{subfigure}

\caption{Plot of the behavior of the two proposed algorithms on four datasets. algoalt has been repeated 100 times following a multi-start strategy and in two settings T=20 (blue) and T=100 (in orange). Each point on the orange and blue lines reports the cumulative time and best objective found during these 100 restarts. PartLS-Opt (PartLS-BnB) outputs a single solution, drawn in red  (green).}
\label{fig:time-vs-obj-figures}

\end{figure}

\end{document}
